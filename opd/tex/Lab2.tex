\documentclass[12pt]{article}
\usepackage{graphicx} % Required for inserting images


\title{1}
\author{Вячеслав Кулагин}
\date{October 2023}
\pagestyle{empty}
\usepackage[utf8]{inputenc}
\usepackage[russian]{babel}
\usepackage[top=1cm,left=1cm,right=2cm, bottom=2cm]{geometry}
\usepackage{tabularx}
\usepackage{amsmath, xparse}
\usepackage{amssymb}
\usepackage{multirow}
\pagestyle{plain}


\begin{document}

\begin{titlepage}
\newpage

\begin{center}
федеральное государственное автономное образовательное
учреждение
высшего образования «Национальный исследовательский
университет ИТМО»\\
\end{center}

\vspace{8em}

\begin{center}
\Large Факультет программной инженерии и компьютерной техники \\ 
\end{center}

\vspace{2em}

\begin{center}
\textsc{\textbf{Лабораторная работа по ОПД №2}}\\
\Large Вариант 6543
\end{center}

\vspace{18em}



\newbox{\lbox}
\savebox{\lbox}{\hbox{Саржевский Иван Анатольевич}}
\newlength{\maxl}
\setlength{\maxl}{\wd\lbox}
\hfill\parbox{11cm}{
\hspace*{5cm}\hspace*{-5cm}Студент:\hfill\hbox to\maxl{Кулагин Вячеслав Дмитриевич\hfill}\\
\hspace*{5cm}\hspace*{-5cm}Преподаватель:\hfill\hbox to\maxl{Саржевский Иван Анатольевич}\\
\\
\hspace*{5cm}\hspace*{-5cm}Поток:\hfill\hbox to\maxl{1.9}\\
}


\vspace{\fill}

\begin{center}
Санкт-Петербург \\2023
\end{center}
\end{titlepage}

\section{Задание}
По выданному преподавателем варианту определить функцию, вычисляемую программой, область представления и область допустимых значений исходных данных и результата, выполнить трассировку программы, предложить вариант с меньшим числом команд. При выполнении работы представлять результат и все операнды арифметических операций знаковыми числами, а логических операций набором из шестнадцати логических значений.\\
\textbf{Исходные данные для варианта 6543:}\\
\includegraphics[scale=1]{lab2.png} 
\newpage
\section{Процесс выполнения работы}
\subsection{Текст исходной программы в виде таблицы}

\texttt{
\begin{tabular}{l|c|c|c|l|}
\cline{2-5}
 & \textbf{Адрес} & \textbf{Код команды} & \textbf{Мнемоника} & \textbf{Комментарии} \\ \cline{2-5} 
 & 142 & E14F & $-$ & Данные $-$ переменная a \\ \cline{2-5} 
 & 143 & 0280 & $-$ & Данные $-$ итоговый результат (b) \\ \cline{2-5} 
 & 144 & E14F & $-$ & Данные $-$ переменная c \\ \cline{2-5} 
\multicolumn{1}{c|}{\textbf{+}} & 145 & 0200 & CLA & 0 $\rightarrow$ AC (Обнуление аккумулятора) \\ \cline{2-5} 
 & 146 & 0280 & NOT & $\overline{AC} \rightarrow AC$ (Отрицание значения в аккумуляторе) \\ \cline{2-5} 
 & 147 & 2142 & AND 142 & 142 \& AC $\rightarrow$ AC (\& между аккумулятором и ячейкой 142) \\ \cline{2-5} 
 & 148 & 214E & AND 14E & 14E \& AC $\rightarrow$ AC (\& между аккумулятором и ячейкой 14E) \\ \cline{2-5} 
 & 149 & E14F & ST 14F & AC $\rightarrow$ 14F (Перемещение из аккумулятора в ячейку 14F) \\ \cline{2-5} 
 & 14A & A144 & LD 144 & 144 $\rightarrow$ AC (Перемещение из ячейки 144 в аккумулятор) \\ \cline{2-5} 
 & 14B & 414F & ADD 14F & 14F + AC $\rightarrow$ AC (Сложение ячейки 14F с аккумулятором) \\ \cline{2-5} 
 & 14C & E143 & ST 143 & AC $\rightarrow$ 143 (Перемещение из аккумулятора в ячейку 143) \\ \cline{2-5} 
 & 14D & 0100 & HLT & Останов \\ \cline{2-5} 
 & 14E & 2142 & $-$ & Данные $-$ переменная d \\ \cline{2-5} 
 & 14F & 414F & $-$ & Данные $-$ переменная e (временное хранение)\\ \cline{2-5} 
\end{tabular}}

\subsection{Описание программы}
Общую формулу описанной программы можно записать так (исключив лишние переносы из ячейки в ячейку и аккумулятор):\\
\begin{large}
c $+$ (1 \& a \& d)\\
\end{large}
Однако она по сути может быть сокращена до:\\
\begin{large}
c $+$ (a \& d)\\
\end{large}
Результат выполнения записывается в ячейку 143\\
Таким образом, данные располагаются в ячейках: 142, 143, 144, 14E, 14F\\
А команды в ячейках: 145, 146, 147, 148, 149, 14A, 14B, 14C, 14D\\
Первой выполняется команда в ячейке 145, последней $-$ в 14D

\subsection{Таблица трассировки} 
\begin{center}
\texttt{
\begin{tabular}{|cc|cccccccc|cc|}
\hline
\multicolumn{2}{|c|}{\textbf{\begin{tabular}[c]{@{}c@{}}Выполняемая\\ команда\end{tabular}}} & \multicolumn{8}{c|}{\textbf{\begin{tabular}[c]{@{}c@{}}Содержание регистров\\ после выполнения команды\end{tabular}}} & \multicolumn{2}{c|}{\textbf{\begin{tabular}[c]{@{}c@{}}Ячейка, содержание\\ которой поменялось\end{tabular}}} \\ \hline
\multicolumn{1}{|c|}{Адрес} & Код & \multicolumn{1}{c|}{IP} & \multicolumn{1}{c|}{CR} & \multicolumn{1}{c|}{AR} & \multicolumn{1}{c|}{DR} & \multicolumn{1}{c|}{SP} & \multicolumn{1}{c|}{BR} & \multicolumn{1}{c|}{AC} & N Z V C & \multicolumn{1}{c|}{Адрес} & Новый код \\ \hline
\multicolumn{1}{|c|}{145} & 200 & \multicolumn{1}{c|}{146} & \multicolumn{1}{c|}{0200} & \multicolumn{1}{c|}{145} & \multicolumn{1}{c|}{0200} & \multicolumn{1}{c|}{000} & \multicolumn{1}{c|}{0145} & \multicolumn{1}{c|}{0000} & - Z - - & \multicolumn{1}{c|}{-} & - \\ \hline
\multicolumn{1}{|c|}{146} & 280 & \multicolumn{1}{c|}{147} & \multicolumn{1}{c|}{0280} & \multicolumn{1}{c|}{146} & \multicolumn{1}{c|}{0280} & \multicolumn{1}{c|}{000} & \multicolumn{1}{c|}{0146} & \multicolumn{1}{c|}{FFFF} & N - - - & \multicolumn{1}{c|}{-} & - \\ \hline
\multicolumn{1}{|c|}{147} & 2142 & \multicolumn{1}{c|}{148} & \multicolumn{1}{c|}{2142} & \multicolumn{1}{c|}{142} & \multicolumn{1}{c|}{E14F} & \multicolumn{1}{c|}{000} & \multicolumn{1}{c|}{0147} & \multicolumn{1}{c|}{E14F} & N - - - & \multicolumn{1}{c|}{-} & - \\ \hline
\multicolumn{1}{|c|}{148} & 214E & \multicolumn{1}{c|}{149} & \multicolumn{1}{c|}{214E} & \multicolumn{1}{c|}{14E} & \multicolumn{1}{c|}{2142} & \multicolumn{1}{c|}{000} & \multicolumn{1}{c|}{0148} & \multicolumn{1}{c|}{2142} & - - - - & \multicolumn{1}{c|}{-} & - \\ \hline
\multicolumn{1}{|c|}{149} & E14F & \multicolumn{1}{c|}{14A} & \multicolumn{1}{c|}{E14F} & \multicolumn{1}{c|}{14F} & \multicolumn{1}{c|}{2142} & \multicolumn{1}{c|}{000} & \multicolumn{1}{c|}{0149} & \multicolumn{1}{c|}{2142} & - - - - & \multicolumn{1}{c|}{14F} & 2142 \\ \hline
\multicolumn{1}{|c|}{14A} & A144 & \multicolumn{1}{c|}{14B} & \multicolumn{1}{c|}{A144} & \multicolumn{1}{c|}{144} & \multicolumn{1}{c|}{E14F} & \multicolumn{1}{c|}{000} & \multicolumn{1}{c|}{014A} & \multicolumn{1}{c|}{E14F} & N - - - & \multicolumn{1}{c|}{-} & - \\ \hline
\multicolumn{1}{|c|}{14B} & 414F & \multicolumn{1}{c|}{14C} & \multicolumn{1}{c|}{414F} & \multicolumn{1}{c|}{14F} & \multicolumn{1}{c|}{2142} & \multicolumn{1}{c|}{000} & \multicolumn{1}{c|}{014B} & \multicolumn{1}{c|}{0291} & - - - C & \multicolumn{1}{c|}{-} & - \\ \hline
\multicolumn{1}{|c|}{14C} & E143 & \multicolumn{1}{c|}{14D} & \multicolumn{1}{c|}{E143} & \multicolumn{1}{c|}{143} & \multicolumn{1}{c|}{0291} & \multicolumn{1}{c|}{000} & \multicolumn{1}{c|}{014C} & \multicolumn{1}{c|}{0291} & - - - C & \multicolumn{1}{c|}{143} & 0291 \\ \hline
\multicolumn{1}{|c|}{14D} & 0100 & \multicolumn{1}{c|}{14E} & \multicolumn{1}{c|}{0100} & \multicolumn{1}{c|}{14D} & \multicolumn{1}{c|}{0100} & \multicolumn{1}{c|}{000} & \multicolumn{1}{c|}{014D} & \multicolumn{1}{c|}{0291} & - - - C & \multicolumn{1}{c|}{-} & - \\ \hline
\end{tabular}}
\end{center}



\end{document}